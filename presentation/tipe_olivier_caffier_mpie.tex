%%%%%%%%%%%%%%%%%%%%%%%%%%%%%%%%%%%%%%%%%%%%%%%%%%%%%%%%%%%%%%%%%%%%%%%
%
%   TIPE OLIVIER CAFFIER MPI(*) Faidherbe
%
%%%%%%%%%%%%%%%%%%%%%%%%%%%%%%%%%%%%%%%%%%%%%%%%%%%%%%%%%%%%%%%%%%%%%%%

\documentclass{beamer}

\usetheme[hideothersubsections]{UNLTheme}
\usepackage{tikz}
\usepackage{xcolor}
\usepackage{algorithm}
\usepackage{algpseudocode}
\usepackage{listings}
\usepackage{multicol}
\definecolor{marine_blue}{RGB}{3, 4, 94}
\lstset{
    framerule=1pt,
    frame=tb,
    emphstyle={\small\ttfamily\bfseries\color{Orange}},
    numbers=left,
    numberstyle= \tiny\color{black},
    basicstyle = \small\ttfamily,
    keywordstyle    = \bfseries\color{red},
    identifierstyle = \bfseries\color{violet},
    stringstyle     = \bfseries\color{black},
    commentstyle    = \bfseries\color{olive},
    breaklines      =   true,
    columns         =   fixed,
    basewidth       =   .5em,
    backgroundcolor=\color{violet!10},
    tabsize=2,
    showspaces=false,
    showstringspaces=false,
}


\title{TIPE }
\author{} %
\institute{}
\date{Résolution algorithmique et génération de problèmes}

\begin{document}

%{% open a Local TeX Group
%\setbeamertemplate{sidebar}{}
\begin{frame}
        \titlepage
		\begin{center}
			\color{marine_blue}\textbf{Go-getter}
		\end{center}
\end{frame}
%}% end Local TeX Group


\section{Introduction}

\begin{frame}
    \frametitle{Introduction}
    \framesubtitle{Le jeu en lui-même}
    \begin{center}
    	\includegraphics[scale=0.3]{images/plateau_photo_2.jpg}
    \end{center}
   

\end{frame}
\begin{frame}
    \frametitle{Introduction}
    \framesubtitle{Le jeu en lui-même}
    \begin{center}
    	\includegraphics[scale=0.3]{images/plateau_photo_3.jpg} 
    \end{center}
   

\end{frame}

\begin{frame}
    \frametitle{Introduction}
    \framesubtitle{Le jeu en lui-même}
    \begin{center}
    	\includegraphics[scale=0.3]{images/plateau_photo_4.jpg} 
    \end{center}
   

\end{frame}

\begin{frame}
    \frametitle{Introduction}
    \framesubtitle{Le jeu en lui-même}
	24 problèmes différents : 
	\begin{center}
		\hspace*{\fill} \includegraphics[scale=0.04]{images/plateau_photo_12.jpg}  \hspace*{\fill} \includegraphics[scale=0.14]{images/plateau_photo_10.jpg} \hspace*{\fill} \includegraphics[scale=0.04]{images/plateau_photo_11.jpg} \hspace*{\fill}
	\end{center}
    

\end{frame}
\begin{frame}
    \frametitle{Introduction}
    \framesubtitle{Le jeu en lui-même}
	Chemins parfaits, imparfaits 


	\begin{center}
		\hspace*{\fill}
		\includegraphics[scale=0.6]{images/bruteforce_screen_3} \hspace*{\fill}
		\includegraphics[scale=0.18]{images/plateau_photo_1.jpg} \hspace*{\fill}
	\end{center}
    

\end{frame}
\begin{frame}
	\frametitle{Introduction}
	\framesubtitle{Brève étude du problème}
	\begin{center}
		\includegraphics[scale=0.12]{images/denombrement.jpeg}
\end{center}
Soit $n$ le nombre de pièces :
	\begin{center}
		$\underbrace{n!}_{n \textit{ emplacements}} \times \underbrace{4^{n}}_{ \textit{rotations}}$ plateaux possibles 
	\end{center}
	\small environ $10^{11}$ configurations différentes.
	
\end{frame}


\section{Modélisation}    

\begin{frame}
    \frametitle{Modélisation}
    \framesubtitle{Une pièce quelconque, du jeu au schéma (1)}
		Prenons une pièce du jeu, qu'on nommera $P_1$ : 
			\begin{center}
				\includegraphics[scale=0.25]{images/p4}
				\vskip 0.25cm
				\includegraphics[scale=0.25]{images/p4_modif1}
				\vskip 0.25cm
				\includegraphics[scale=0.3]{images/p4_modif2}
				
			\end{center}
\end{frame}

\begin{frame}
    \frametitle{Modélisation}
    \framesubtitle{Une pièce quelconque, du jeu au schéma (2)}
			\begin{center}
				\includegraphics[scale=0.4]{images/p4_modif2}
			\end{center}
			
				On convertit alors cette pièce par une liste d'adjacence : 
				\begin{enumerate}
					\setcounter{enumi}{-1}
					\item 
					\item $[2 ; 3 ]$
					\item $[1 ; 3 ]$
					\item $[1 ; 2 ]$
				\end{enumerate}
				
				
			
\end{frame}

\begin{frame}
	\frametitle{Modélisation}
	\framesubtitle{Une pièce quelconque, du schéma au code}
		\begin{enumerate}
					\setcounter{enumi}{-1}
					\item 
					\item $[ 2 ;  3 ]$
					\item $[ 0; 3 ]$
					\item $[0 ; 2]$
		\end{enumerate}
		et on convertit cette liste d'adjacence en C : 
		\includegraphics[scale=0.8]{images/piece_quelconque_implementation}
\end{frame}


	
	\begin{frame}
		\frametitle{Modélisation}
		\framesubtitle{Moteur graphique}
			\begin{center}
				\includegraphics[scale=0.9]{images/modelisation.png}
			\end{center}
	\end{frame}
	
	


\begin{frame}
	\frametitle{Modélisation}
	\framesubtitle{Vocabulaire}
		On définit le vocabulaire suivant : 
		\begin{itemize}
			\item \textit{Deck} : ensemble de cartes
			\item \textit{Slot} : case (remplie/non-remplie) du plateau disposant des côtés numérotés (de la même manière qu'une pièce) de $0$ à $3$.
		\end{itemize}
\end{frame}

\begin{frame}
	\frametitle{Modélisation}
	\framesubtitle{Implémentation d'un deck}
	
	Le jeu nous met à dispositions $9$ cartes, on va les placer dans un tableau (que l'on nommera l'\textit{index}), un peu à la manière de mettre à plat sur une table toutes les cartes : 
	\vspace*{\fill}
	\begin{center}
		\vspace*{\fill}
		\includegraphics[scale=0.5]{images/used_cards_v1} \vspace*{\fill} \\
		\includegraphics[scale=0.2]{images/plateau_photo_5.jpg}
	\end{center}
	\vspace*{\fill}
	
	
\end{frame}

\begin{frame}
	\frametitle{Modélisation}
	\framesubtitle{Définition des types}
	\small On définit alors les types suivants : 
	\includegraphics[scale=0.5]{images/types_implementation}


	\end{frame}
	
	
\section{Bruteforce}
\begin{frame}
	\frametitle{Bruteforce}
	\framesubtitle{Avec des schémas (0)}
	Posons en guise d'exemple deux conditions $C_1$ et $C_2$ avec : 
	\begin{itemize}
		\item $C_1$ : \includegraphics[scale=0.01]{/Users/sacss/Desktop/Prépa/6.TIPE/Code/pictures/white_mouse.png} $\rightarrow$ \includegraphics[scale=0.01]{/Users/sacss/Desktop/Prépa/6.TIPE/Code/pictures/red_cheese.png}
		\item $C_2$ : \includegraphics[scale=0.01]{/Users/sacss/Desktop/Prépa/6.TIPE/Code/pictures/grey_mouse.png} $\rightarrow$ \includegraphics[scale=0.01]{/Users/sacss/Desktop/Prépa/6.TIPE/Code/pictures/white_cat.png}
	\end{itemize}
	\begin{center}
		\includegraphics[scale=0.12]{images/exemple_bruteforce.png}
	\end{center}
\end{frame}

\begin{frame}
	\frametitle{Bruteforce}
	\framesubtitle{Avec des schémas (1)}
	\begin{center}
		\includegraphics[scale=0.5]{images/bruteforce_screen_1}
	\end{center}
	\footnotesize
	Choisissons $P_1$ comme première pièce posée, sans rotation : 
	
	\begin{center}
		\includegraphics[scale=0.7]{images/used_cards_v2}
	\end{center}
	
\end{frame}

\begin{frame}
	\frametitle{Bruteforce}
	\framesubtitle{Avec des schémas (2) }
		\begin{center}
			\includegraphics[scale=0.5]{images/bruteforce_screen_2}
		\end{center}
	\footnotesize
	Choisissons ensuite $P_2$ comme deuxième pièce posée, sans rotation : 
	
	\begin{center}
		\includegraphics[scale=0.7]{images/used_cards_v3}
	\end{center}
	
		
\end{frame}

\begin{frame}
	\frametitle{Bruteforce}
	\framesubtitle{Avec des schémas (3)}
	\begin{center}
			\includegraphics[scale=0.5]{images/bruteforce_screen_3}
		\end{center}
	\footnotesize
	 Remplissons le reste... On obtient un plateau qu'on note $\mathbb{T}_1$
	
	\begin{center}
		\includegraphics[scale=0.7]{images/used_cards_v5}
	\end{center}
\end{frame}

\begin{frame}
	\frametitle{Bruteforce}
	\framesubtitle{Avec des schémas (4) }
	Traduisons la condition \color{red}$C_1$\color{black} sur le schéma.
	\begin{center}
		\includegraphics[scale=0.7]{images/bruteforce_screen_4}
	\end{center}
\end{frame}

\begin{frame}
	\frametitle{Bruteforce}
	\framesubtitle{Avec des schémas (5)}
	On voit bien que $\mathbb{T}_1$ respecte bien \color{red}$C_1$\color{black}
		\begin{center}
			\includegraphics[scale=0.7]{images/bruteforce_screen_5}
		\end{center}
\end{frame}

\begin{frame}
	\frametitle{Bruteforce}
	\framesubtitle{Avec des schémas (6)}
	Néanmoins \color{orange}$C_2$ \color{black}  est insatisfaite par $\mathbb{T}_1$
		\begin{center}
			\includegraphics[scale=0.7]{images/bruteforce_screen_6}
		\end{center}
\end{frame}


\begin{frame}
	\frametitle{Bruteforce}
	\framesubtitle{Avec des schémas (7)}
	\begin{center}
		\includegraphics[scale=0.4]{images/bruteforce_screen_6} \\ $\Downarrow$  \\ \includegraphics[scale=0.1]{bruteforce_screen_15}
	\end{center}
	\footnotesize
	On modifie la dernière pièce posée \textbf{en faisant une rotation de 90 degrés} sur cette dernière.
	
\end{frame}

\begin{frame}
	\frametitle{Bruteforce}
	\framesubtitle{Avec des schémas (8)}
	Néanmoins ce nouveau plateau ne respecte pas $C_2$. On réitère alors notre opération de rotation…
	\begin{center}
		\includegraphics[scale=0.3]{images/bruteforce_screen_15} \\
		\includegraphics[scale=0.3]{images/bruteforce_screen_14} \\
		\includegraphics[scale=0.3]{images/bruteforce_screen_15}
		
	\end{center}
\end{frame}
\begin{frame}
	\frametitle{Bruteforce}
	\framesubtitle{Avec des schémas (9)}
	Aucune rotation ne correspond, on retourne à l'avant dernier slot :
	\begin{center}
		\includegraphics[scale=0.3]{images/bruteforce_screen_11} \\
		\includegraphics[scale=0.3]{images/bruteforce_screen_12} \\
		\includegraphics[scale=0.3]{images/bruteforce_screen_13}
		
	\end{center}

\end{frame}
\begin{frame}
	\frametitle{Bruteforce}
	\framesubtitle{Avec des schémas (10)}
	Etc… On recommence en faisant tourner $P_1$ de 90 degrés : 
	\begin{center}
		\includegraphics[scale=0.6]{images/bruteforce_screen_16.png}
	\end{center}

\end{frame}

\begin{frame}
	\frametitle{Bruteforce}
	\framesubtitle{Avec des schémas (11)}
	\begin{center}
		\includegraphics[scale=0.7]{images/bruteforce_screen_8}
	\end{center}
	On obtient alors un plateau $\mathbb{T}_2$
\end{frame}

\begin{frame}
	\frametitle{Bruteforce}
	\framesubtitle{Avec des schémas (12)}
	\begin{center}
		\includegraphics[scale=1]{images/bruteforce_screen_9}
	\end{center}
	Et on remarque que $\mathbb{T}_2$ respecte bien  $C_1$  et  $C_2$
\end{frame}

 \begin{frame}
	\frametitle{Bruteforce}
	\framesubtitle{Étude de complexité}
	%\includegraphics[scale=0.33]{pseudo_code_1}
	Complexité : 
	\begin{center}
		$\mathcal{O}(n! \times 4^n)$ 
	\end{center}

	\begin{flushright}
		il est d'ailleurs NP-Complet ($\leq_p$ Couverture exacte). 
		D'où le fait qu'on reste sur la stratégie backtrack par la suite, en l'optimisant.
	\end{flushright}

	
\end{frame}

\section{Optimisation}
	

\begin{frame}
	\frametitle{Optimisation}
	\framesubtitle{Premier problème : éviter de compléter linéairement}
	Prenons le plateau avec les contraintes suivantes : 
	\begin{center}
		\includegraphics[scale=0.18]{images/premier_probleme.png}
	\end{center}
\end{frame}
\begin{frame}
	\frametitle{Optimisation}
	\framesubtitle{Premier problème : éviter de compléter linéairement}
	Le joueur humain prendra le réflexe de d'abord compléter les cases avec les contraintes : 
	\begin{center}
		\vspace*{\fill}
		\includegraphics[scale=0.2]{images/plateau_photo_6.jpg} 
		\includegraphics[scale=0.2]{images/plateau_photo_7.jpg}  \\ \vspace*{\fill} 

	\end{center}
\end{frame}

\begin{frame}
	\frametitle{Optimisation}
	\framesubtitle{Premier problème : éviter de compléter linéairement}
	On va donc créer une file de priorité pour trouver les slots prioritaires à traiter et les ordonner
	\begin{itemize}
		\item On identifie le nombre de contraintes qu'elle possède 
		\item On lui attribue une valeur de priorité en fonction de ce nombre
	\end{itemize}

\end{frame}

\begin{frame}
	\frametitle{Optimisation}
	\framesubtitle{Premier problème : éviter de compléter linéairement}
	On associe la priorité $20$ par exemple au slot à 2 contraintes : 
	\begin{center}
		\includegraphics[scale=0.3]{images/opti_screen_2.png}
	\end{center}
	et la priorité $10$ aux deux autres slots : 
	\begin{center}
		 \includegraphics[scale=0.3]{images/opti_screen_3.png} \\  \includegraphics[scale=0.3]{images/opti_screen_4.png} 
	\end{center}
	\begin{flushright}
		Les autres slots possèdent la priorité 0. On sait donc quels sont les slots à traiter en premier.
	\end{flushright}
\end{frame}

\begin{frame}
	\frametitle{Optimisation}
	\framesubtitle{Troisième problème : gestion du deck}
	Concernant la file \textbf{par slot}, 
	\begin{itemize}
		\item On sélectionne juste les cartes qui sont capables de répondre au nombre de contraintes
	\end{itemize}
	Par exemple, pour la contrainte suivante, il n'y a que les cartes à au moins 2 chemins qui peuvent convenir : 
	\begin{center}
		\includegraphics[scale=0.3]{images/opti_screen_2.png}
	\end{center}
	\begin{flushright}
		ce qui permet de réduire le nombre de cartes éligibles par slot, et donc réduire le nombre de coups.
	\end{flushright}
\end{frame}


\begin{frame}
	\frametitle{Optimisation}
	\framesubtitle{Deuxième problème : vérification partielle des configurations}

	Il y a également un besoin de répondre directement si une configuration partielle correspond au critère ou non. \\ 
	Exemple d'une configuration posant problème : 
	\begin{center}
		\includegraphics[scale=0.4]{images/opti_screen_5.png}
	\end{center}
\end{frame}
\begin{frame}
	\frametitle{Optimisation}
	\framesubtitle{Deuxième problème : vérification partielle des configurations}
	On va donc créer une heuristique vérifiant si : 
	\begin{enumerate}
		\item la construction respecte bien les contraintes imposées pour les cartes déjà posées (exemple précédent)
		\item il n'y a pas de cycles fermés
		\item il n'y pas de situations dites sans issues (profondeur de recherche limitée à 1)
	\end{enumerate}
\end{frame}

\begin{frame}
	\frametitle{Optimisation}
	\framesubtitle{Deuxième problème : vérification partielle des configurations}
	2. Exemple de cycle fermé
	\begin{center}
		\includegraphics[scale=0.5]{images/cycle_ferme.jpeg}
	\end{center}
\end{frame}

\begin{frame}
	\frametitle{Optimisation}
	\framesubtitle{Deuxième problème : vérification partielle des configurations}
	3. Exemple de situation sans issue 
	\begin{center}
		\includegraphics[scale=0.8]{images/sans_issue.png}
	\end{center}
	
	
\end{frame}
\begin{frame}
	\frametitle{Bruteforce}
	\framesubtitle{Quelques problèmes, performance}
	

	Problèmes : 
	\begin{enumerate}
		\item Complète linéairement le tableau 
		\item Pas d'élimination immédiate des configurations absurdes, impossibles
		\item Dépend trop de l'ordre dans lequel les cartes sont disposées sur le deck 
	\end{enumerate}
	
	Statistiques : 
	\begin{center}
		ne peut résoudre que les niveaux faciles en moins d'1'
	\end{center}
	
\end{frame}
\begin{frame}
	\frametitle{Optimisation}
	\framesubtitle{Bilan des optimisations}
	Bilan de performance (arrondie à l'unité supérieure): 
	\begin{center}
		\begin{tabular}{|c||c|c|}
			\hline
			Niveaux & Sans opt, c.i& Sans opt, c.p \\ \hline
			Facile &  1' &  30' \\ \hline
			Moyen & 15' & $\infty$  \\ \hline
			Difficile & $\infty$ & $\infty$  \\ \hline
			Très Difficile & $\infty$ & $\infty$ \\ \hline
		\end{tabular}
	\end{center}
	\begin{center}
		\begin{tabular}{|c||c|c|}
			\hline
			Niveaux &  Avec opt, c.i & Avec opt, c.p \\ \hline 
			Facile &  5" & 10" \\ \hline
			Moyen &  8" & 15" \\ \hline
			Difficile &  1' & 4' \\ \hline
			Très Difficile &  5' & 30'  \\ \hline
		\end{tabular}

	\end{center}
	\footnotesize c.i = chemin imparfait ; c.p = chemin parfait ; opt = optimisation 
\end{frame}

\section{Génération de problèmes}

\begin{frame}
	\frametitle{Génération de problèmes}
	\framesubtitle{Quelques niveaux}
	Générer un problème à $k$ contraintes consiste à la réalisation de 2 étapes : 
	\begin{itemize}
		\item Générer aléatoirement $k \times $ (2 emplacements distincts et leurs côtés)
		\item Vérifier avec notre algorithme si une solution au problème généré avec chemin parfait existe. On déduit le niveau de difficulté du temps requis à l'algorithme.
	\end{itemize}
	
\end{frame}
\begin{frame}
	\frametitle{Génération de problèmes}
	\framesubtitle{Quelques niveaux}
	Un exemple de problème généré et résolu : 
	\begin{multicols}{2}
		\begin{itemize}
			\item \includegraphics[scale=0.01]{/Users/sacss/Desktop/Prépa/6.TIPE/Code/pictures/bone_1.png} $\rightarrow$ \includegraphics[scale=0.01]{/Users/sacss/Desktop/Prépa/6.TIPE/Code/pictures/bone_2.png}
			\item \includegraphics[scale=0.01]{/Users/sacss/Desktop/Prépa/6.TIPE/Code/pictures/white_cat.png} $\rightarrow$ \includegraphics[scale=0.01]{/Users/sacss/Desktop/Prépa/6.TIPE/Code/pictures/grey_dog.png}
			\item \includegraphics[scale=0.01]{/Users/sacss/Desktop/Prépa/6.TIPE/Code/pictures/white_mouse.png} $\rightarrow$ \includegraphics[scale=0.01]{/Users/sacss/Desktop/Prépa/6.TIPE/Code/pictures/bone_1.png}
			\item \includegraphics[scale=0.01]{/Users/sacss/Desktop/Prépa/6.TIPE/Code/pictures/tree.png} $\rightarrow$ \includegraphics[scale=0.01]{/Users/sacss/Desktop/Prépa/6.TIPE/Code/pictures/bone_1.png} \includegraphics[scale=0.01]{/Users/sacss/Desktop/Prépa/6.TIPE/Code/pictures/bone_2.png}
			\item \includegraphics[scale=0.01]{/Users/sacss/Desktop/Prépa/6.TIPE/Code/pictures/fox.png} $\rightarrow$ \includegraphics[scale=0.01]{/Users/sacss/Desktop/Prépa/6.TIPE/Code/pictures/yellow_cheese.png}
		\end{itemize}

		\begin{center}
			\includegraphics[scale=0.24]{images/plateau_photo_1.jpg}
		\end{center}
	\end{multicols}
	
\end{frame}



\end{document}

